\documentclass[a4paper, 11pt]{article} % Font size (can be 10pt, 11pt or 12pt) and paper size (remove a4paper for US letter paper)

\usepackage{../essay}
\usepackage{pdfpages}
\graphicspath{ {../img/}{../../img/} }

\begin{document}

\textit{This student nomination for the Dean’s Award for Excellence in Mentoring has been prepared by Brooks Mershon (B.S. Duke University). It is accompanied by a supporting excerpt from an essay written in August 2016, which attempts to describe the experience of taking Chris’s \textit{Digital 3D Geometry} course and its impact on the author.}

\bigskip

What I hope to impress upon the committee is the positive, lasting effect that working with Chris has had on my life. I happened to meet Chris first when he was a graduate student visiting Paul Bendich's course titled \textit{Topology with Applications} in 2015, and then one year later as my instructor (still a graduate student) for an ambitious course titled \textit{Digital 3D Geometry}, taught as part of a Bass Family Fellowship during the last semester of my undergraduate career. Chris was first an older student, then an instructor, and then a friend to me. I enrolled in his course solely based on the impression he made on me during his hour-long guest lecture in Bendich’s course.


I think a mentor is someone who you want to make proud with the work that you do, indeed who you become. Chris has been just such a guiding light as my studies came to an end at Duke and I have begun new projects, new studies, and new work as a software engineer working at SketchUp after graduation. More than any other student or faculty member at Duke, Chris has raised the standard of how I hope to pursue my future academic endeavors, and perhaps more importantly, how I will work with other people, how I will teach other people. Chris inspired and will continue to inspire me in every sense of the word: I see in him the characteristics of the problem solver, writer, teacher, and friend that I want to become. And all this because he encouraged me over the course of a semester to do my very best work throughout a series of really cool and challenging projects. Chris has a great attitude, and I like to think it has rubbed off on me.


Chris is a gifted problem solver and an articulate instructor. Viewed as a teacher, he is as thoughtful as they come. I have watched him over the course of an intense semester hone his ability to spot problem areas before students encounter them, quickly incorporating feedback received from classroom message boards and student responses to create an effective two-way conversation in the classroom. His enthusiasm is contagious, encouraging myself and my peers in his course to step up to the demanding and challenging nature of the significant programming assignments undertaken throughout a semester. To put it simply, Chris’s course rewarded people who backed talent with work ethic, where perhaps many may have been accustomed to gliding by without truly investing their own time and energy in the pursuit of learning. Chris brings his background as a hacker and maker turned mathematician to bare on the particular pedagogical flavor of his course. Work in his course (and I imagine any future course he will teach) centers around authorship, around spending inordinate amounts of time with your head down so that you may emerge with fantastic, creative, and working evidence of your own understanding of the material. Chris puts a premium on hard work and encourages people to exceed expectations because, well, who doesn't want to have their work receive the top prize in the "art gallery" that accompanied each assignment.


Chris is willing to embrace the experimental as an instructor (certainly as a PhD student as well!), and his ability and willingness to be open with his communication (in person and online) allowed his students to, from what I could see, work harder and longer than they were accustomed to in other courses where rote memorization, rather than creativity and \textit{doing}, was the normal mode of learning a rigorous subject. There is a risk of students falling out of lockstep with such an audacious curriculum, especially when the complexity of programming assignments threatened to obscure and derail learning the fundamentals. Chris navigates this challenge with communication. And it works. To my mind, Chris effortlessly models a beautiful and positive attitude towards learning fueled by intellectual curiosity and pride in one’s work, rather than a fear of poor performance on, say a single exam. This classroom model that Chris creates is a natural consequence of his pure attitude towards his own research and projects. 


Throughout the semester, Chris’s lectures are routinely supplemented with brief (and not so brief) discussions of intellectual integrity and lessons learned from his own experiences, both good and bad, of working with others in an academic environment where egos, fear, and even more toxic emotions plague efforts to do great work and teach others along the way. Suffice it to say, to the surprise (and delight) of Chris’s students, lessons often danced between the particular material (say of computational geometry or a high-level overview of topology) and the general strategies of learning and persisting when things get difficult along a course of study.


These supplementary discussions were lessons drawn from encounters he has had with advisers and his own peers. Discussions which left him feeling disgusted, even angry with what he was hearing, were relayed so that we might receive some warning that other instructors usually wholly leave out of an undergraduate education. Lessons were drawn from his own personal struggle with balancing intense academic work with his own life. It is in such sojourns outside of the technical stuff we were learning into the areas of “how to be a young adult and work with others do do interesting things without going crazy” that I realized why I like Chris so much, and why he has the capacity to be a great mentor. Chris has a finely tuned ability to feel out the psychology and emotions surrounding learning (and the institutions associated with the study of things like advanced mathematics). Chris cannot help but empathize with his students, and he certainly cannot help stopping them in their tracks to try to help them when he spots an opportunity. He understands learning is an emotional journey. I've seen first hand how well he does with both the most advanced students and those who find themselves struggling. Certain instances in which I performed less than admirably in navigating group dynamics on a given assignment provided Chris an opportunity to gently correct my course. 


I feel Chris was a mentor to an entire classroom of students by simply showing up and spreading his contagious enthusiasm. I know Chris set a high standard of work ethic by bending over backwards to produce engaging assignments with thousands of lines of boilerplate code he had written which meshed perfectly with the resources he curated for us. Slides, papers, videos, diagrams, and any other materials students contributed throughout the course or that he had written immediately before a particular lecture, amended to reflect prior progress. That this effort took perhaps four or five times more time than most faculty members might invest in a course was not lost on us students. Nor was the fact that he was, in fact, still a student. In return, I feel it was only natural that even the less motivated among us would spend those extra ten hours over a weekend to see a project through to completion. It’s curious how much pride was reflected in my projects for his course compared to the work I had done in over twenty other courses at Duke. It rarely felt like work. It felt like an opportunity to prove we understood things; for Chris to tell us that once again, the class had blown him away with the work they turned in.


But the reason Chris was a particularly special mentor to me was because of the discussions that took place in a one-on-one setting, usually during office hours. My enthusiasm for the course and my concerns as I drew near graduation and the dread of not yet having found a “fulfilling” job presented me with a lot of questions I wanted to ask and only one person I felt could help me. I was both excited by the immediate work I was doing and anxious as I encountered pangs of aimlessness. There were a lot of things I needed to reason through by talking with someone who had for the most part "been there, done that." Chris was generous with his time for such soul-searching. I'm sure because he was doing a bit of that himself. He was just a bit further along than I was.


One half of Chris's role as a mentor for me was that of the academic coach, where every interaction only fired me up to spend more time doing neat things with the big final project I had set for myself (visualizing the cutting up of one polygon into another of equal area). I so looked forward to trudging across campus to hike up the flight of stairs in Gross Hall, hoping I would be the only student to show up so that Chris and I could excitedly review what I had built and brainstorm the next step to take, as he too was eager to see what my partner and I would build. I wanted to continually exceed his expectations because he knew just the right way to set another goal a few steps further and let me go off to chase it. When other students happened to show, I waited my turn, and Chris was equitable with his time.


The other half of Chris’s role as a mentor was admittedly more along the lines of a psychologist's. For better or worse, Chris recognized in me certain problems that he himself wrestles with. Sometimes I felt paralyzed by doing great work, only to come up with an impossibly long list of areas of study I did not know and felt I ought to know. “Don’t let not knowing everything stop you from doing anything,” he once said to me over an email when I started fielding questions about a supplementary curriculum I was attempting to set up for myself. He, too, has done things he was proud of only to turn around and grow anxious when he started looking at other folks who knew things he did not and felt he ought to know.


Such was this other half of Chris’s mentoring: the role in which he helped me navigate emotional and mental blocks to the whole enterprise of learning. Ego, fear, anxiety, competition, fulfillment, relationships, cooperation, and integrity. All these things were hashed out in bits and pieces as we thundered down an exciting course together. He was the teacher, and I was the student, but we would not have spent so many hours at the whiteboard together or hours talking in the cafeteria if it were not for us both thinking that we had something to learn from the other. As with every mentor I have had who I feel invested a great deal in me, there was a balance of both give and take. For all the advice and encouragement Chris had to offer me, I had encouragement and thoughts to give back as well. This too, is what makes makes Chris an excellent mentor. He is always trying to learn; he does this by reaching out to people so that he may both share and listen. I know this attitude will take him far. I hope to be a mentor to others as Chris continues to be for me.


\end{document}