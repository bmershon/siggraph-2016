%%%%%%%%%%%%%%%%%%%%%%%%%%%%%%%%%%%%%%%%%
% Thin Sectioned Essay
% LaTeX Template
% Version 1.0 (3/8/13)
%
% This template has been downloaded from:
% http://www.LaTeXTemplates.com
%
% Original Author:
% Nicolas Diaz (nsdiaz@uc.cl) with extensive modifications by:
% Vel (vel@latextemplates.com)
%
% License:
% CC BY-NC-SA 3.0 (http://creativecommons.org/licenses/by-nc-sa/3.0/)
%
%%%%%%%%%%%%%%%%%%%%%%%%%%%%%%%%%%%%%%%%%

%----------------------------------------------------------------------------------------
%	PACKAGES AND OTHER DOCUMENT CONFIGURATIONS
%----------------------------------------------------------------------------------------

\documentclass[a4paper, 11pt]{article} % Font size (can be 10pt, 11pt or 12pt) and paper size (remove a4paper for US letter paper)
\usepackage{subfiles}
\usepackage{essay}
\graphicspath{ {img/}{../img/} }

%----------------------------------------------------------------------------------------
%	TITLE
%----------------------------------------------------------------------------------------

\title{\textbf{SIGGRAPH 2016}\\ % Title
A wide-eyed account} % Subtitle

\author{\textsc{Brooks Mershon}} % Author

\date{\today} % Date

%----------------------------------------------------------------------------------------

\begin{document}

\maketitle % Print the title section

\eject

%----------------------------------------------------------------------------------------
%	ABSTRACT AND KEYWORDS
%----------------------------------------------------------------------------------------

\renewcommand{\abstractname}{Forward} % Uncomment to change the name of the abstract to something else

\section*{Forward}
	
\textbf{S}pecial \textbf{I}nterest \textbf{G}roup on computer \textbf{GRAPH}ics (and Interactive Techniques). The following is based on the experience of a 22 year old Student Volunteer (SV) attending the 43rd SIGGRAPH Converence on Computer Graphics and Interactive Techniques for the first time. The conference was held in the Anaheim Convention Center from 24 July to 28 July, 2016. Following graduation and the beginning of an exciting new job working on the popular CAD program \textit{SketchUp}, this SV booked a ticket from Boulder to Anaheim, checked in with an Airbnb host, and joined a band of other accepted SVs from across the country and overseas who had all been tasked with ensuring that SIGGRAPH 2016 runs smoothly (and having fun in the process).

This is my little story about how I heard of SIGGRAPH, why I applied to be a student volunteer, how I got there, what it was like, and what I learned. This story spirals out a bit further from a faithful log of the events an atendee's perspective: it furrows down veins covering an attendee's undergraduate experience and mindset both going into and actually experiencing the event. Afterall, SV stands for \textit{Student} Voluntee; it is during a new graduate's transition from university to industry that SIGGRAPH took place. In fact, SIGGRAPH is a kind of celebration of this milestone, at least for the author, and attending the event confers certain special privileges and unique experiences that apply to these newly minted or soon-to-be-minted graduates schools like Ringling, RISD, Carleton, SCAD, Carnegie Mellon, SAIC, Dartmouth, Texas A\&M, NHTV Breda University of Applied Sciences (Netherlands), The University of Edinburgh. A lot of schools, a lot of  Shared experiences. It's my story, but I would not have minded reading such an account from another student when I was in my senior year of college, trying to evaluate all of the information and opportunties demanding my attention.

There are three purposes my account of SIGGRAPH 2016 should serve.	First, it should let you know what SIGGRAPH is so you can decide whether or not the conference is something you would like to attend in one capacity or another. To be an SV, you must be accepted, and to attend at any one of several regular tiers, you must be willing to pony up the cash. Getting accepted isn't necessarily hard. Second, the following should shed some light on the rhythm of the event, hopefully highlighting for other participants---especially other SVs---aspects of the conference which they might like to focus on. This includes what goes on \textit{around} the conference. We're in Anaheim, for better or worse; SISGGRAPH 2017 will be in Los Angeles. SIGGRAPH is a deluge of information, so selectivity is a necessity. Each day comes and screams past. There are a lot of people, myriad booths, and jam-packed programming navigable by kiosk and convient App Store guides, as well as the event program every SV will have tucked into their badge hanging around their neck.

This is not a list of dos and don'ts, but rather a description of what I managed to take in while I was an attendee, which was certainly affected by my role, my housing accomodations, and the current stage of my education and career. This account lets me reflect on why this young whippersnapper went to an academic conference on graphics during the summer after graduating from Duke. That's certainly worthwhile for me and you are welcome to skip around or stop reading now. This is for me as much as it is for you, the person from whose perspective you will be engaging what follows.

\eject

\tableofcontents
%----------------------------------------------------------------------------------------
%	ESSAY BODY
%----------------------------------------------------------------------------------------

\eject

\subfile{sections/prologue}
\newpage
\subfile{sections/arrival}
\newpage
\subfile{sections/day_0}
\newpage
\subfile{sections/day_1}
\newpage
\subfile{sections/day_2}
\newpage
\subfile{sections/day_3}
\newpage
% \subfile{sections/05}
\newpage
\subfile{sections/departure}

\newpage


%------------------------------------------------

%----------------------------------------------------------------------------------------
%	BIBLIOGRAPHY
%----------------------------------------------------------------------------------------

%\bibliographystyle{unsrt}

%\eject 

%\bibliography{references}

%----------------------------------------------------------------------------------------

\end{document}