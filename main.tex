%%%%%%%%%%%%%%%%%%%%%%%%%%%%%%%%%%%%%%%%%
% Thin Sectioned Essay
% LaTeX Template
% Version 1.0 (3/8/13)
%
% This template has been downloaded from:
% http://www.LaTeXTemplates.com
%
% Original Author:
% Nicolas Diaz (nsdiaz@uc.cl) with extensive modifications by:
% Vel (vel@latextemplates.com)
%
% License:
% CC BY-NC-SA 3.0 (http://creativecommons.org/licenses/by-nc-sa/3.0/)
%
%%%%%%%%%%%%%%%%%%%%%%%%%%%%%%%%%%%%%%%%%

%----------------------------------------------------------------------------------------
%	PACKAGES AND OTHER DOCUMENT CONFIGURATIONS
%----------------------------------------------------------------------------------------

\documentclass[a4paper, 11pt]{article} % Font size (can be 10pt, 11pt or 12pt) and paper size (remove a4paper for US letter paper)
\usepackage{subfiles}
\usepackage{essay}
\usepackage{pdfpages}
\usepackage{ulem} % strikethrough macro
\graphicspath{ {img/}{../img/} }

%----------------------------------------------------------------------------------------
%	TITLE
%----------------------------------------------------------------------------------------

\title{\textbf{SIGGRAPH 2016}\\ % Title
A wide-eyed account} % Subtitle

\author{\textsc{Brooks Mershon}} % Author

\date{August 15, 2016} % Date

%----------------------------------------------------------------------------------------

\begin{document}

\maketitle % Print the title section

\eject

%----------------------------------------------------------------------------------------
%	ABSTRACT AND KEYWORDS
%----------------------------------------------------------------------------------------

\renewcommand{\abstractname}{Forward} % Uncomment to change the name of the abstract to something else

\section*{Forward}
	
\textbf{S}pecial \textbf{I}nterest \textbf{G}roup on computer \textbf{GRAPH}ics (and Interactive Techniques). The following is based on the experience of a 22 year old Student Volunteer (SV) attending the 43rd SIGGRAPH Conference on Computer Graphics and Interactive Techniques for the first time. The conference was held in the Anaheim Convention Center from the 24th through the 26th of one particularly hot week in July. Following graduation and the beginning of an exciting new job working on the CAD program \textit{SketchUp}, this SV booked a ticket from Boulder to Anaheim, checked in with an Airbnb host, and joined a band of other accepted SVs from across the country and overseas who had all been tasked with ensuring that SIGGRAPH 2016 runs smoothly (and having fun in the process).

This is my little story about how I heard of SIGGRAPH, why I applied to be a student volunteer, how I got there, what it was like, and what I learned. This story spirals out a bit further from a faithful log of the events: it furrows down veins covering an attendee's undergraduate experience and mindset both before and during the event. SV stands for \textit{Student} Volunteer; it is during this new graduate's transition from university to industry that SIGGRAPH took place. In fact, SIGGRAPH is a kind of celebration of this milestone, at least for the author, and attending the event confers certain special privileges and unique experiences that apply to these newly minted or soon-to-be graduates from schools like Ringling, RISD, Carleton, SCAD, Carnegie Mellon, UNC, SAIC, Dartmouth, Texas A\&M, NHTV Breda University of Applied Sciences (Netherlands), and The University of Edinburgh. A lot of schools, a lot of shared experiences. Several of the students, like me, had just graduated and were working at places like Apple, MPC, and Dreamworks. Many SVs were returning for another year of volunteering and the vast majority were here in large part for the job opportunities. This my story, but I would not have minded reading such an account from another student when I was in my senior year of college, trying to evaluate all of the information and opportunities demanding my attention.

This is not a list of dos and don'ts, but rather a description of what I managed to take in while I was an attendee, which was certainly affected by my role, my housing accommodations, and the current stage of my education and career. I have tried to just throw down on the page a blow-by-blow of events with some attention to the thoughts I had as they happened. Some of the thoughts were deep, and others were along the lines of \textit{this was totally awesome} or \textit{that kind of sucked, so I did another thing instead}.

What follows has allowed me to reflect on why this young whippersnapper went to an academic conference on computer graphics during the summer after graduating from Duke. Perhaps what I ate and where we drank and whom I talked to is not that important. Perhaps my account is a little self-indulgent. This was certainly worthwhile for me to write and you are welcome to skip around or stop reading now. This is for me as much as it is for you, the person from whose perspective you will be reading about what went on at SIGGRAPH 2016.

\eject

\tableofcontents
%----------------------------------------------------------------------------------------
%	ESSAY BODY
%----------------------------------------------------------------------------------------

\eject

\subfile{sections/prologue}

\newpage

\subfile{sections/arrival}

\newpage

\subfile{sections/day_0}

\newpage

\subfile{sections/day_1}

\newpage

\subfile{sections/day_2}

\newpage

\subfile{sections/day_3}

\newpage

\subfile{sections/day_4}

\newpage

\subfile{sections/day_5}

\newpage


%------------------------------------------------

%----------------------------------------------------------------------------------------
%	BIBLIOGRAPHY
%----------------------------------------------------------------------------------------

%\bibliographystyle{unsrt}

%\eject 

%\bibliography{references}

%----------------------------------------------------------------------------------------

\end{document}