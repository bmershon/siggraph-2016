\documentclass[../main.tex]{subfiles}
 
\begin{document}

\section{Emerging Tech}

Today is your first shift as a student volunteer. You show up to check in with your Team Leader, who will be responsible for placing you on the Experience Hall floor with a vendor. You express an interest in haptic technology as your pack of student volunteers is being lead throughout the showroom by your leader, so you are checked off on the clipboard for your shift and introduced to some researchers from the Koji-Lab. This lab is based out of the University of Electro-Communications in Tokyo, Japan. Quickly, you realize your chief responsibilities will be greeting attendees who walk by, encouraging them to try their technology while explaining the gist of the pamphlets being handed out. The researchers do not speak english very well; you become the voice face of their product.

What they have brought to the convention center is a device which attempts to produce sensations of texture as you move your virtual hand through a Unity simulation. Before you can begin explaining the demonstration to attendees, you must try it out for yourself. The researchers carefully tuck your thumb, index, and middle finger into sensors in the way a finger pulse oximeter is usually clampled to a person's digits. In front of you is a monitor showing a rendered wooden desk covered with a book, a metal grate with ribs and spirals, a Poké Ball, a pencil, a translucent ball, and a wooden cube next to a square hole. You see an outstretched hand that moves with yours as you move a mouse designed to hold your hand palm down with fingers outstretched, as if ready to hover over the virtual objects on the virtual desk without picking them up.

After calibrating the electro-mechanical setup attached to your fingers, you begin moving your hand (and your virutal fingers) about the scene, the high and low vibrations and small electrical current tricking your fingers into sensations of texture. Though the sensations are odd and do not correspond with any real-world texture, they do permit relative differences in perceived textures among the various items on the virtual desk. You can feel differences between the book's cover and the metal grate as your hand glides over it; the latter in fact exerts patterns of sensations which do suggest invisible "currents" and gradients which have orientations in space similar to what one imagines one would feel if they were accutely sensitive to ambient electrical fields produced by household appliances and mobile computing devices.

For the next three hours, you describe to attendees patiently awaiting their turn to sit down and have the \textit{Fingar} project strapped to their hand that what they will experience is a sensation similar to what one feels when their arm falls asleep---except on their fingertips. You emphasize the relative differences, rather than absolute fidelity to real world textures that they may find particularly interesting.

Near the end of your shift, you notice one gray-haired gentleman in his forties standing near display, scanning the materials on the desk and reading all of the provided display literature. You ask him if he would like to try it out, as you have repeated to hundreds of attendees before him.

"So its electrical and mechanical stimulation?"

Yes, you say, and you confirm the use of low frequency, high frequency, electrical stimulation, and mechanical pressure.

"Hmm. I thought electrical stimulation had gone out of favor."

You look at his badge and see it says Apple, Inc. on it. He proceeds during his demonstration to prod the engineers and requests various switches to be turned on and off on their arduino as he carefully assesses the simulation and inquires about their device. The researchers comply, and you think to yourself that you are watching a gruff Apple hardware engineer prepare to eat this lab's lunch, the gracious researchers unwittingly revealing much of their devices specifications. This probably happens pretty often.

After your shift ends, you check the SIGGRAPH app on your phone and find a Pixar Rendering talk to attend. This is your first talk of the event, held in an upstairs conference room that will host many technical talks throughout the week from companies like Pixar, Autodesk, little boutique animation shops, and Nvidea. While waiting for the show to start, you note stuck by the culture that seems to accompany various companies. For example, the Nvidiea guys all seem uniformly robust and smart looking, overwhelmingly German and more often than not dressed in all black with designer jeans. Perhaps these folks are not representative, but its an interesting pattern, nontheless. The Pixar guys are decidely less intimidating, often donning cool red tracksuit sweatshirts with PIXAR written across the back, and often, unlike the Nvidia guys, wearing shorts.

During the Pixar talk, you are treated to live demos of the Universal Scene Description  (USD) pipeline workflow for managing tremendous numbers of objects and literally billions of triangles. You watch a scene from the animated film \textit{Finding Dory} rendered in Presto in real-time, where fine-grained effects can be layered in to provide animators with a great sense of effects like subdivision surfaces, caustics, and depth of field in real-time. The phrase you hear said again and again is \textit{blasting triangles}. The talk impresses upon you the exciting nature of pipeline engineering, for the shear amount of data and assets in a film liek \textit{Cars} or \textit{Finding Dory} necessitates clever ways of chopping up information and allowing for collaboration and thoughtful management of instances, all of which may not necessarily need to be worked on at a given time.

One of the most impressive examples involved a rigged Lightning McQueen from \textit{Cars} whose face was manipulated in real-time in front of the audience. It was pointed out how riggers need only operate on a smaller number of points than the ones produced in real-time by subdivision surfacing. You witness millions of triangles being blasted onto your screen as the person on stage contorts Ligtning McQueen's face into a hilarious pout. "I'm not an animator," he says as the audience laughs.



\end{document}