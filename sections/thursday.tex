\documentclass[../main.tex]{subfiles}
 
\begin{document}

\section{Wednesday}

\subsection{Finding Dory}

Today is the rough day. Where your previous days saw either one or two shifts totalling less than six hours of time spent repeating, greeting, or seating attendees, today's schedule has been changed since you last checked it on Monday morning. At noon you begin back-to-back shifts in VR Village, followed by a short break before you are staffed to assist in the second showing of the Computer Animation Festival reels. This is a bummer, because you already saw the Computer Animation Festival's first showing on Monday night. Your third shift today will force you to miss the Pixar RenderMan Party for which you had already RSVP'd and expected to joing your new friends in attending. This is the only time this week you have wished to swap shifts, but realize you must respect the 24-hours-notice period during which you are allowed to swap. This third shift had showed up probably 30 hours ago. Surprise.

Acknowledging you would miss most of the programming for Wednesday, you scour your pamphlet once you have at the convention center to see if you can catch anything before your shift starts.  You see that you can at the very least catch the first 45 minutes of a Production Session held in the large Hall B: \textit{Under the Sea: Making of Finding Dory}. These 45 minutes turn out to be some of the most illuminating of your entire week.

Unlike the other technical talks hosted by Nvidia and Autodesk, this session avoided burying the audience in technical details without narrative nor motivation. The Pixar team was seated on stage so that different members of the team could come to the podium and talk as slides presented their contributions to the film. You are most fascinated by a presentation on the set design for Finding Dory. The gentleman speaking brings your attention to his first slide, which shows arrays of panels containing early sketches for various aspects of the film.

The four panels show a circle, a small dot, a squigly line, and a straight line. It is explained that these four panels were a first sketch of four aspects of the set one will experiene along the course of Dory's adventure. They are an attempt to capture emotions in basic forms: the circle represents the organic coral; the dot represents the vastness of the ocean and the vulnerability of a fish isolated and small in the frame; the squigle represents the kelp forest, which provides protection yet still feels alien and ominous; the straight line represents the hard lines of the human world, more alien, and tied to the set that will be built for the Marine Life Institute (MLI). This is fascinating and a revelation for you, because this is your first glimpse behidn the scenes that has exposed a logical process that you could imagine employing in your own work. The playfullness of this design process is fascinating. The next panels show iterations on these first four forms. Color is introduced: the kelp forest turns green; the institute evolves into a network of overlapping lines which turn at sharp angles much like the piping of the institute's vast duct network in the movie.

\begin{quotation}
[The] story's needs and camera angles determine how it's made.
\end{quotation}

When you meet incredibly talented people from places like Pixar, you're intial reaction may be one of fear---you feel insecurity and awe. But quickly you allow yourself to be inspired and set out on reading, studying, doing.

Another slide shows a timeline of emotions in which a line traces emotions on the vertical axis against the progression of the movie's scenes on the horizontal axis. Another abstract tool from Pixar aimed at teasing out the most fundamental essence of the story telling process. It strikes you that this is the kind of thought process lacking in the Experience Hall, where many companies desperately try to jump on the virtual reality bandwagen without investing any serious effort into distilling a story. These 10 minutes devoted to the set design of \textit{Finding Dory}, more than anything else you have seen at SIGGRAPH, reveal the difference between putting emotion and story first and letting technology distract. You recall something you heard during Sunday's \textit{Real-time Graphics for Film Production at Pixar}. Pixar seeks any technological advancement that can help animators tell better stories. It's all about story telling. You knew this, but SIGGRAPH has shown plenty of work that loses sight of this ultimate goal, and the difference between what companies like Pixar and Dreamworks show and that of the other studios is made clear. 

Following the set designer's presentation, a lead texture and materials designer explains the physics of the characters' material properties which lend, for example, the gummy texture to Nemo's exterior under a variety of lighting scenarios. On screen, a simple diagram explains the phenomenon of diffusiuon scatter, and it is explained that physically unrealistic parameters are set on nearly every surface to force gummy and translucent characters when light would indeed not behave the way you are seeing it behave on screen. Part of this texture discussion dives into Pixar's Universal Scene Description method for managing assets and scenegraphs. It is pointed out that adopting USD and the new raytracing paradigm set forth by RenderMan RIS was a big decision for Pixar at approximately 2 years out from the film's release. It is pointed out that nobody knows what RIS stands for, how about "Really Interesting Story," the presenter jokes. From the RenderMan site:

\begin{quote}
The RIS technology in RenderMan is a game-changing rendering paradigm, a highly-optimized framework for rendering global illumination, specifically for ray tracing scenes with heavy geometry, hair, volumes, and irradiance with world-class efficiency in a single pass. This evolution in technology offers best-of-class in rendering for both VFX and feature film animation. Today RenderMan is the most flexible, powerful, and reliable tool for rendering cinematic imagery.  
\end{quote}

In short, RIS is a raytracer that is really good at making scenes look like those found in \textit{Finding Dory}. The rendering technology stack discussion brings up another issue of recovering assets from \textit{Finding Dory}---digital archeology.

Checking the time constantly, you reluctantly get up in time to head for your shift and leave as a directory explains the technical difficulties involved in animating the mischievious, tetacled Hank. You think as you are leaving Hall B that not being good at drawing is an insecurity you would like to dispel over the coming year.

\subsection{Computer Animation Festival}

Worn down from a day spent putting VR headsets on people's heads, taking them off, cleaning them, and putting them on again, you grab another Starbucks Bistro box and sit dazed in the SV office with the other volunteers, watching groups head off to the RenderMan party at another hotel that you will not be venturing to this evening. In an hour, you begin to usher a couple thosuand attendees into the Computer Animation Festival's second showing from 8pm to 10pm. You figure this isn't the worst thing to happen, as you loved the first showing and your scan for recording devices during the show will let you watch most of the shorts again---some of these shorts are not even available outside this evening's showing. Realizing that the event was overstaffed, you are informed part way through the show that you are free to leave or take a seat and watch the rest of the Electronic Theater. It's too late to head to the RenderMan party, and you would honestly like to see some of Disney and Pixar's shorts again near the end of the show. You find a seat next to a girl you've taken a liking to and enjoy the rest of the show. On Monday she was staffing the show, so both of you are seeing most of the shorts for the second time, and it is in watching things for a second time that you can appreciate the particular elements you cherished in each short.

Tomorrow you plan to head to Huntington Beach after the day's shifts and programming have finished with three other friends you have joined on previous nights at receptions and local bars. You retire early to do laundry and get a good night of sleep before the next day's festivities.

\end{document}