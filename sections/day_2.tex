\documentclass[../main.tex]{subfiles}
 
\begin{document}

\section{Computer Animation Festival}

In the morning, you attempt to sit through Nvidia's Vulcan presentation. You meet a coworker seated behind you and settle in for a technical talk which you hope will introduce you to some GPU fundamentals. You are mistaken. Instead, you are treated to slides average 200 words, not counting the ones which literally show hundreds of lines of illegible source code. You are presented with diagram after unintelligble diagram and excuse yourself after 40 minutes of torture. You head down to VR village after wondering about Experience Hall. On your first day at SIGGRAPH, you worked with the Japanese haptic technology lab. Now you would like to experience Emerging Technology (E-Tech) from an attendees perspective.

You try the HapticWave deomonstration at the Oculus booth, which is stationed next to the booth you stationed the previous day. You feel it is wholly more convincing and polished than the work done by the Japanese lab. You then wander about the rest of the Experience Hall, growing increasingly jaded with the new technologies. You decide what you want out of SIGGRAPH are technical talks pitched at an appropriate level and the opportunity to hear behind the scenes talks from the folks who know how to tell stories. Again, the technology fest in E-Tech is simply overwhelming and does not seem to be worth investing your time in.

You do a stressful shift working with Rolland, during which you handle a constant stream of requests to print emailed images onto tote bags. You gain an appreciation for what working in retail must be like and are humbled by how difficult you found the 3 hour rush to process the print request with attendees standing behind you, managing your Adobe Illustrator work. You feel badly at your behavior that a tiny crack appeared in your dike of diplomacy when after a lot of nagging from one attendee, you simply scooted your chair back and pointed at your keyboard, allowing her to draw up her free design as she wished.

Following this shift, you are preparing to get a ticket for the Monday Computer Animation Festival. When you went to the Student Volunteer office in the morning to check in for yoru shifts, you inquired about getting your free ticket. Unfortunatlely, you are informed that only a Wednesday ticket is available. You take the ticket and hope to swap with another student, making a desparate post to the Facebook group asking for a swap. By the time the showtime rolls around, you intend to try your luck at following your friend in line and hoping the student volunteer checking tickets doesn't bother to read the date on your ticket. You are not this lucky when you are denied entrance by a fellow student volunteer at the entrance. After waiting in line for 20 minutes, you are instructed to leave the line that is now about 1,000 deep to run to the opposite end of the convention center to swap your Wednesday ticket for a Monday ticket. You eventually make it inside and, hilariously, end up being ushered to a seat in the second row. This will turn out to be one of the most visually impressive places to experience the show, as well as the most accoustically traumatic. The bass literally blows your hair around throughout the show. You find this both uncomfortable and totally awesome.

The Computer Animation Festival is truly a special event. Lasting from 6pm to 8pm, you are treated to award-winning student films, reels by companies like MPC showcasing their technological prowess, and finally shorts by Disney and Pixar. Disney's short, \textit{Inner Workings} is being exclusively shown at SIGGRAPH, so you are eager to see a glimpse at shorts which are not avaiable to the genral public. The full rundown:

\begin{enumerate}
	\item \textit{Accidents, Blunders and Calamities} --- a hilarious depiction of inventive ways for animals and insects to meet their demise.
	\item \textit{Tea Time} --- A grandma and her robot. Very French and a bit odd in its choice of style.
	\item \textit{Alike} --- One of several shorts aimed at showing how much work and failing to march to the beat of your own drum can be a real drag.
	\item \textit{Tokyo Cosmo} - A dreamlike sequence for a young girl living in Japan.
	\item \textit{Glass Half} --- Absolutely steals the show with a brilliant 2D cartoon using nonsensical vocalizations. Beautiful and funny.
	\item \textit{Citipati} --- A dinosaur and a meteor.
	\item \textit{Natural Attraction} --- The drama of a vocanic eruption. I felt chills watching this.
	\item \textit{League of Legends: Project Overdrive} --- The art style was bananas.
	\item \textit{Solar Superstorms} --- An example of animation based off fantastically complex simulations. This demonstration emphasizes a need for real-time data-visualization for simulation that is presented at a Nvidia talk later in the week.
	\item \textit{Mafia III} --- Video games are getting pretty (gruesome).
	\item \textit{Lichtspiel} --- A mindblowing intro sequence for a film festival .
	\item \textit{Terminator Genisys} --- MPC showing off.
	\item \textit{Behind The Magic: Warcraft} --- Industrial Light \& Magic Showing off.
	\item \textit{Les marmottes: Mariachi} --- Hilariously placed between other longer shorts.
	\item \textit{Behind The Magic: Marvel's Captain America: Civil War} --- Layered effects.
	\item \textit{Crabe-Phare} --- Best Student Project award winner and my absolute favorite film that wasn't Pixar's \textit{Piper}.
	\item \textit{Shell V-Power Shapeshifter} --- A really involved commercial.
	\item \textit{Borrowed Time} --- One of those films that raises the bar for story-telling in animation. Winner of Best in Show.
	\item \textit{Escargore} --- Slugs. Funny in the way \textit{Accidents, Bludners and Calamities} was refreshing and full of physical comedy.
	\item \textit{Cosmos Laundromat} --- The soundtrack alone gave me chills. This is part of continued project that is oddly released under Creative Commons by the Blender Foundation.
	\item \textit{Taking Flight} --- Good like \textit{Toy Story} is good.
	\item \textit{Moana} --- Sneak peek at an upcomming Disney film.
	\item \textit{None of that} --- Ringling College of Art and Design.
	\item \textit{Inner Workings} --- I've got a poster of this short at my desk at work.
	\item \textit{Piper} --- So cute. The rendering is as mind blowing as the elegance of the story.
\end{enumerate}

\end{document}