\documentclass[../main.tex]{subfiles}
 
\begin{document}

\section{Application}
 
During your last semester of college, you happen upon a course that will be taught by a graduate student whose name you recognize: the previous year, Chris Tralie delivered a guest lecture to you and a dozen or so other students enrolled in an introductory course in computational topology. That survey of topology was the first of several positive academic experiences that occured during the half of your undergraduate career. Not knowing very much at all about the field of topology and certainly lacking the mathematical background to cope with a pure presentation of topology, you nonetheless found yourself at home in a course designed to sweep much of the mechanics under the rug, leaving the big ideas out in the open to impress, inspire, and possibly encourage subsequent, deeper sudy.

In this topology course taken the previous year, you had the chance to \textit{do math} for the first time at a chalkboard alongside a teacher willing to help; you talked and drew diagrams in order to play fast and loose with new abstractions as you began stepping away from learning math the rote way towards learning through (visual) intuition. A light bulb went off, to say the least. In that course, you had the opportunity to experience new models of teaching---new ways for you to learn. You created interactive presentations of the concepts illustrated in class and experienced the joy of learning math by making things---indeed learning math by placing it alongside \textit{design} and \textit{programming}. A bit of math anxiety seems to drop away behind you. It's still there, but you have more positive associations with math than you did before. You wrote code to visualize ideas and derived tremendous satisfaction from seeing that work. If that's what math is, or what it can be, you think you ought to forge ahead, even after you graduate. Maybe math is design, or design is math. You can't help being drawn towards math you can \textit{show} to others after you receive praise for your final project. As it turns out, the whole enterprise of working on math you can show to others is something in which Chris Tralie has invested a great deal of time.

The spirit of that introduction to topology: \textit{learn by doing, learn by applying}. The course revolved around a significant final project in lieu of an exam and somewhere in the middle of the semester, a fourth year graduate student named Chris Tralie presented a lecture that tied into an exploratory exercise he had crafted for the class. \textit{Data Expeditions}, it was called. The lab required students to use Matlab and a special visualization tool he had created to examine the structure of songs. The structure of songs visualized by his software presented itself as lines looping around in three dimensional space; the bridge of a song often corresponded to a quick dash from one loop over to another loop in space. The loops and undulations formed in real-time as the song played and more than fifty dimensions were projected down to our easily visualized three. A tremendous amount of effort went into designing this exercise and it is Chris's thoughtfulness and playfulness in his approach to pedagogy, demonstrated during his 50 minute lecture and anwsers to students' questions over email, that left a positive impression of him as a teacher.

That Chris was not all that much older than you made it all the more easy to imagine someday having the exciting opportunity to teach something as he was doing. Your visualization tools and the presentations you made of their use and development let you try on the hat of the instructor---the presenter---and you rather liked doing that. Pacing back and forth in front of the class, sweating in the humid math building during the Fall as he excitedly painted a picture of his work and the applications of our course to his own research, Chris was clearly one of those graduate students who actually wanted to teach and would only get better at it. He was raring for the opportunity to share his unique insights into digital signal processing, computational geometry, and the ways he has had to build his own mathematical scaffolding as he paved his way through academia. His whole life he has been a hacker. He was now coming into his own as a teacher as well. He was 26 when he first stopped in for the guest lecture and it is no surprise that he was awarded a teaching fellowship the following year, allowing him to function as the instructor for a course of over 40 students, decked out with mirrored projectors, whiteboards, and a fancy new classroom suitable for a highly visual math course.

When you see that Chris is teaching \textit{Digital 3D Geometry} in the Spring of your senior year, you check to see if CS290 will fit in your schedule. It will, but it will be your fifth course, usually considered overloading at your university. You show up for the second class and talk to the instructor after class, inquiring about auditing the course\textellipsis
 
\end{document}