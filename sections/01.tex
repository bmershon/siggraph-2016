\documentclass[../main.tex]{subfiles}
 
\begin{document}

\section{Second semester senior year}

\subsection{Enter Chris Tralie}
 
During your last semester of college, you enroll in a course that will be taught by student whose name you recognize: during the Fall of the previous year, a graduate student named Chris Tralie delivered a guest lecture to you and about a dozen other students enrolled in an introductory course in computational topology. That survey of topology taken during your junior year was the first of several positive academic experiences that occured during the second half of your undergraduate career.

Not knowing very much at all about the field of topology and certainly lacking the mathematical background to cope with a pure presentation of the subject matter, you nonetheless found yourself relatively comfortable in a course designed to sweep much of the mechanics under the rug. The big ideas were left out in the open to impress, inspire, and possibly encourage subsequent and deeper sudy. In fact, most of these big ideas were communicated at the chalkboard as diagrams. You found them inuitive and beautiful. In a way, the reason you are going to attend SIGGRAPH 2016 is that you took MATH 412 with Paul Bendich during the Fall of 2015. That course introduced you to both ideas and people from whom you would learn a lot about, well, learning. That course led to instructors who become friends, to glowing letters of recommendation, to internships, and to your ticket to SIGGRAPH. That course eventually led a job. MATH 412 changed your attitude about math, learning, and communication.

In this topology course, you had the chance to \textit{do math} for the first time at a chalkboard alongside a teacher with a gift for explication. You talked during office hours and drew diagrams in order to play fast and loose with new abstractions as you began stepping away from learning math the rote way towards engaging unfamiliar territory with whatever machinery you needed to build up your intuitions. For you, diagrams had a tremendous attraction. Diagrams---lines, dashes, points, commutative relations with arrows and spatial cues---seemed less tedius than symbolic manipulation as you had previously encountered it in the form of algebraic relations and proofs with which you had little experience or foothold. Bendich was still feeding you some abstract stuff, but the way he made it dance and the narratives that accompanied the \textit{math} made it not only palatable, but enjoyable. A light bulb went off, to say the least. In MATH 412, you had the opportunity to experience new models of teaching. This wasn't because the material in MATH 412 was so special, but rather because the instructor was just that good at briding the gap between his own mental models and the tools that a student may or may not have to work with. He knew when things didn't click, and he knew detours that others did not ever try which could bring the confused out of the dark. You've said to others about your experiences in MATH 412 that Bendich is a new breed of teacher---the nouvau-instructor. He was 36 and that number would come to sound higher than you would have expected. His excellence in teaching has only been matched by individuals who were 28 and 27 years old at the time they were your instructors.

In MATH 412 you were learning about a geometry unconcerend with exact dimensions and entirely concerned with how components are connected to one another. This is important, because as abstract as that could become, therein lay the wiggle room to allow someoone not on the \textit{Math Track} to play and create and grow. The best language for topological initiation was probably symbols if you had the right backgorund. The second best was a neat drawing on the chalkboard. You wondered if you could expand on this visual mode of learning by leveraging your modest programming experience enthusiasm for Mike Bostock's data-visualization work to bring these chalkboard diagrams to life. You figured data-visualization made a lot of sense in topology if you let the data be geometry. You developed interactive presentations of the concepts illustrated in class and experienced the joy of learning math by making things---indeed learning math by placing it alongside \textit{design} and \textit{programming}. It is the beauty of those chalboard diagrams that you latched onto and decided to bring alive with JavaScript running in the browser. You began putting many hours each week into your vision of a final project that would be delivered as the semester ended in early December.

A bit of math anxiety seems to drop away behind you during your junior year. It's still there, but you have more positive associations with math than you did before taking MATH 412. You wrote code to visualize ideas and derived tremendous satisfaction from seeing that work. If that's what math is, or what it can be, you think you ought to forge ahead, even after you graduate, in your pursuance of math. Maybe math is design, or design can be math. You've never been bad at math, but you've also never found it particularly pleasing or easy. To your friends, you've explained math as something you never felt you could dance in, where all things on the other side of standardized testing you felt came more natural to you. You can't help being drawn towards math you can \textit{show} to others after you receive praise for your final project. As it turns out, the whole enterprise of working on math you can show to others is something in which Chris Tralie has invested a great deal of time.

The spirit of that introduction to topology: \textit{learn by example, learn by applying}. The course revolved around a significant final project in lieu of an exam and somewhere in the middle of the semester, Chris Tralie , then a fourth year graduate student, presented a lecture that tied into an exploratory exercise he had crafted for the class. \textit{Data Expeditions}, it was called. The lab required students to use Matlab and a special visualization tool he had created to examine the structure of songs. The patterns to be found in these songs were visualized by his software as curves winding around in three-dimensional space; the bridge of a song often corresponded to a quick dash from one loop over to another loop, and the curve would often cycle around a loop multiple times before engaging another loop in space. The loops and undulations formed in real-time as the song played. More than fifty dimensions were projected down to our easily visualized three. A tremendous amount of effort went into designing this exercise and it is Chris's thoughtfulness and playfulness in his approach to pedagogy, demonstrated during his 50 minute lecture and anwsers to students' questions over email, that left a positive impression of him as a teacher.

That Chris was not all that much older than you made it easy to imagine someday having the exciting opportunity to teach as he was doing. Your own visualization projects and the presentations you made of their use and development let you try on the hat of the instructor---the presenter---and you rather liked that type of work. You liked thinking about how others would see your work and experience the ideas you are trying to illustrate.

 Pacing back and forth in front of the class, sweating in the humid math building during one of the scheduled lectures for MATH 412, Chris excitedly painted a picture of his work and the applications of your course to his own research. He was raring for the opportunity to share his unique insights into connections between digital signal processing, computational geometry, and his own mathematical scaffolding he has constructed as he paved his way through academia. His whole life he has been a hacker. He was by now coming into his own as a teacher and academic as well. He was 26 when he first delievered the guest lecture to MATH 412 and it is no surprise that he was awarded a teaching fellowship the following year which allowed him to serve as the instructor for a course of over 40 students with a curiculum entirley of his own design.

\subsection{Application} 
 
When you see that Chris is teaching \textit{Digital 3D Geometry} in the Spring of your senior year, you check your clunky online portal to see if CS 290 happens to fit in your gridlocked schedule. It does, but it will be your fifth course, usually considered overloading at your university. You show up for the second lecture and talk to the instructor after class, inquiring about auditing the course. He remembers you and expresses how much he would enjoy having you in his course, reminding you that he would not be able to give you as much attention during office hours to help with projects if you audit. The projects are where the real learning takes place, and you know you are going to want attention and help, so you enroll.

CS 290 quickly becomes your top priority. It is a class in which every lecture is a whirlwind sojourn in an entirely new slice of Chris's reasearch, with small pillars of theoretical fundamentals erected only as they are needed. The first few lectures cover vectors, dot products, and notions of duality in geometry. The next few weeks after that involve matrix operations and a sigificant project on ray tracing and scene graphs. At some point you learn about quaternions. Then you leap to statistics which can be performed on point clouds to categorize shapes. You learn how to do linear algebra quite quickly with Python and you begin to wrestle with high-dimensional data analysis. Eventually, topology creeps up and you encounter the Euler characteristic, followed by a brief intro to data structures for geometry. Near the very end, you learn about Chris's connections made between geometry and his deep study of digital signal process
 
\end{document}