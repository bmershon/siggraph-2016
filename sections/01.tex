\documentclass[../main.tex]{subfiles}
 
\begin{document}

\section{Second semester senior year}

\subsection{Enter Chris Tralie}
 
During your last semester of college, you happen upon a course that will be taught by a graduate student whose name you recognize: the previous year, Chris Tralie delivered a guest lecture to you and a dozen or so other students enrolled in an introductory course in computational topology. That survey of topology was the first of several positive academic experiences that occured during the second half of your undergraduate career. Not knowing very much at all about the field of topology and certainly lacking the mathematical background to cope with a pure presentation of the subject matter, you nonetheless found yourself at home in a course designed to sweep much of the mechanics under the rug, where the big ideas were left out in the open to impress, inspire, and possibly encourage subsequent and deeper sudy. In a way, the reason you are going to attend SIGGRAPH 2016 is that you took MATH 412 with Paul Bendich during the Fall 2015 semester. That course introduced you to both ideas and people from whom you would learn a lot about how you learn. That course led to instructors who become friends, letters of recommendation, internships, a ticket to SIGGRAPH, and eventually a job.

In this topology course, you had the chance to \textit{do math} for the first time at a chalkboard alongside a teacher with a gift for explaining; you talked and drew diagrams in order to play fast and loose with new abstractions as you began stepping away from learning math the rote way towards learning through (visual) intuition. A light bulb went off, to say the least. In MATH 412, you had the opportunity to experience new models of teaching---new ways for you to learn. This wasn't because the material in MATH 412 was so special, but rather because the instructor was just that good. You were learning about a geometry unconcerend with exact dimensions and entirely concerned with how components are connected to each other. You created interactive presentations of the concepts illustrated in class and experienced the joy of learning math by making things---indeed learning math by placing it alongside \textit{design} and \textit{programming}.

A bit of math anxiety seems to drop away behind you during your junior year. It's still there, but you have more positive associations with math than you did before taking MATH 412. You wrote code to visualize ideas and derived tremendous satisfaction from seeing that work. If that's what math is, or what it can be, you think you ought to forge ahead, even after you graduate. Maybe math is design, or design is math. You've never been bad at math, but you've also never found it particularly pleasing or easy. You can't help being drawn towards math you can \textit{show} to others after you receive praise for your final project. As it turns out, the whole enterprise of working on math you can show to others is something in which Chris Tralie has invested a great deal of time.

The spirit of that introduction to topology: \textit{learn by example, learn by applying}. The course revolved around a significant final project in lieu of an exam and somewhere in the middle of the semester, a fourth year graduate student named Chris Tralie presented a lecture that tied into an exploratory exercise he had crafted for the class. \textit{Data Expeditions}, it was called. The lab required students to use Matlab and a special visualization tool he had created to examine the structure of songs. The patterns to be found in these songs were visualized by his software as curves winding around in three-dimensional space; the bridge of a song often corresponded to a quick dash from one loop over to another loop, and the curve would often cycle around a loop multiple times before engaging another loop in space. The loops and undulations formed in real-time as the song played. More than fifty dimensions were projected down to our easily visualized three. A tremendous amount of effort went into designing this exercise and it is Chris's thoughtfulness and playfulness in his approach to pedagogy, demonstrated during his 50 minute lecture and anwsers to students' questions over email, that left a positive impression of him as a teacher.

That Chris was not all that much older than you made it easy to imagine someday having the exciting opportunity to teach as he was doing. Your own visualization projects and the presentations you made of their use and development let you try on the hat of the instructor---the presenter---and you rather liked that type of work. You liked thinking about how others would see your work and experience the ideas you are trying to illustrate.

 Pacing back and forth in front of the class, sweating in the humid math building during one of the scheduled lectures for MATH 412, Chris excitedly painted a picture of his work and the applications of your course to his own research. He was raring for the opportunity to share his unique insights into connections between digital signal processing, computational geometry, and his own mathematical scaffolding he has constructed as he paved his way through academia. His whole life he has been a hacker. He was by now coming into his own as a teacher and academic as well. He was 26 when he first delievered the guest lecture to MATH 412 and it is no surprise that he was awarded a teaching fellowship the following year which allowed him to serve as the instructor for a course of over 40 students with a curiculum entirley of his own design.

\subsection{Application} 
 
When you see that Chris is teaching \textit{Digital 3D Geometry} in the Spring of your senior year, you check your clunky online portal to see if CS 290 happens to fit in your gridlocked schedule. It does, but it will be your fifth course, usually considered overloading at your university. You show up for the second lecture and talk to the instructor after class, inquiring about auditing the course. He remembers you and expresses how much he would enjoy having you in his course, reminding you that he would not be able to give you as much attention during office hours to help with projects if you audit. The projects are where the real learning takes place, and you know you are going to want attention and help, so you enroll.

CS 290 quickly becomes your top priority. It is a class in which every lecture is a whirlwind sojourn in an entirely new slice of Chris's reasearch, with small pillars of theoretical fundamentals erected only as they are needed. The first few lectures cover vectors, dot products, and notions of duality in geometry. The next few weeks after that involve matrix operations and a sigificant project on ray tracing and scene graphs. At some point you learn about quaternions. Then you leap to statistics which can be performed on point clouds to categorize shapes. You learn how to do linear algebra quite quickly with Python and you begin to wrestle with high-dimensional data analysis. Eventually, topology creeps up and you encounter the Euler characteristic, followed by a brief intro to data structures for geometry. Near the very end, you learn about Chris's connections made between geometry and his deep study of digital signal process
 
\end{document}